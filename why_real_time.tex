\section{Why real-time}

\begin{frame}{The Conceptual Advantage of Real-Time Information for Measurement Systems}

\begin{itemize}
\item Real-time information gives users a better understanding of the measurement process and enables them to make more informed decisions based on the data.
\item Real-time information allows users to monitor the measurement process in real time and detect any anomalies or issues as they occur.
\item Real-time information about uncertainties allows users better to understand the reliability and accuracy of the measured data.
\item Real-time information about uncertainties can help users optimize the measurement process itself.
\end{itemize}

\end{frame}

\begin{frame}{Real-Time Information Enables Effective Troubleshooting and Maintenance}

\begin{itemize}
\item Real-time information allows for quick and effective troubleshooting and maintenance, which can help prevent further errors or inaccuracies.
\item Early detection of issues is critical in safety-critical systems where even a small delay in detecting a problem can have serious consequences.
\end{itemize}

\end{frame}

\begin{frame}{Real-Time Information Improves Accuracy and Precision}

\begin{itemize}
\item Real-time information allows users to adjust the measurement process in real-time to improve accuracy and precision.
\item Real-time information about uncertainties allows users to make more informed decisions about how to use the data and what level of confidence they can have in the results.
\item Real-time monitoring of uncertainties can help users optimize the measurement process and reduce uncertainties to improve overall accuracy and precision.
\end{itemize}

\end{frame}

\begin{frame}{Real-Time Information Enables Faster Decision-Making}

\begin{itemize}
\item Real-time output allows for faster decision-making by providing immediate feedback on the measured data.
\item Real-time information is especially important in time-sensitive applications where quick decisions can make a significant difference.
\end{itemize}

\end{frame}

\begin{frame}{Real-Time Information Enables Process Control}

\begin{itemize}
\item Real-time output can be used to monitor and control processes in real time, allowing for adjustments to be made as needed to ensure optimal performance.
\item Real-time information is especially important in manufacturing and industrial applications where process control can impact quality, safety, and efficiency.
\end{itemize}

\end{frame}

\begin{frame}{Real-Time Information Enables Remote Monitoring}

\begin{itemize}
\item Real-time output can enable remote monitoring of measurement systems, allowing users to monitor and control systems from a distance.
\item Real-time information is especially important in applications where access to the measurement system may be limited or difficult.
\end{itemize}

\end{frame}

\subsection{Why real-time 3D-PTV}

\begin{frame}{Advantages of Real-Time 3D-PTV}

\begin{itemize}
\item Real-time 3D-PTV allows for immediate feedback on the velocity field, which can be used for process control and optimization.
\item Real-time 3D-PTV enables faster decision-making by providing immediate feedback on the fluid flow behavior, allowing for quick adjustments to be made as needed.
\item Real-time 3D-PTV can provide detailed insight into the fluid flow behavior, which can be used for research purposes and to improve understanding of fluid dynamics.
\item Real-time 3D-PTV can help improve the accuracy and precision of the measured data by allowing for adjustments to be made in real-time to reduce uncertainties.
\item Real-time 3D-PTV can help detect any anomalies or issues in the fluid flow behavior as they occur, allowing for quick and effective troubleshooting and maintenance.
\item Real-time 3D-PTV can be used to remotely monitor fluid flow behavior, which can be especially important in applications where access to the measurement system may be limited or difficult.
\end{itemize}

\end{frame}

\begin{frame}{Immediate Feedback for Process Control and Optimization}

\begin{itemize}
\item Real-time 3D-PTV allows for immediate feedback on the velocity field, which can be used for process control and optimization.
\item Faster decisions can be made to adjust the process as needed, ensuring optimal performance and efficiency.
\end{itemize}

\end{frame}

\begin{frame}{Faster Decision-Making}

\begin{itemize}
\item Real-time 3D-PTV enables faster decision-making by providing immediate feedback on the fluid flow behavior, allowing for quick adjustments to be made as needed.
\item Faster decisions can help prevent delays and improve overall efficiency in applications where time is critical.
\end{itemize}

\end{frame}

\begin{frame}{Detailed Insight into Fluid Dynamics}

\begin{itemize}
\item Real-time 3D-PTV can provide detailed insight into the fluid flow behavior, which can be used for research purposes and to improve understanding of fluid dynamics.
\item More detailed information can lead to new discoveries and improvements in fluid dynamics research.
\end{itemize}

\end{frame}

\begin{frame}{Improved Accuracy and Precision}

\begin{itemize}
\item Real-time 3D-PTV can help improve the accuracy and precision of the measured data by allowing for adjustments to be made in real-time to reduce uncertainties.
\item Real-time monitoring of uncertainties can help reduce errors and improve confidence in the measured data.
\end{itemize}

\end{frame}

\begin{frame}{Effective Troubleshooting and Maintenance}

\begin{itemize}
\item Real-time 3D-PTV can help detect any anomalies or issues in the fluid flow behavior as they occur, allowing for quick and effective troubleshooting and maintenance.
\item Early detection of issues is critical in safety-critical applications where delays can have serious consequences.
\end{itemize}

\end{frame}

\begin{frame}{Remote Monitoring}

\begin{itemize}
\item Real-time 3D-PTV can be used to remotely monitor fluid flow behavior, which can be especially important in applications where access to the measurement system may be limited or difficult.
\item Remote monitoring can allow for more efficient use of resources and improve safety in hazardous environments.
\end{itemize}

\end{frame}