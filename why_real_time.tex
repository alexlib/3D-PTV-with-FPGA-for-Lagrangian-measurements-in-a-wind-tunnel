
\section{Why real-time 3D-PTV is a good idea}
\topicFramePrimary{Why real-time 3D-PTV? }

\begin{frame}[label=why-1]{Fluid dynamicists: }
    \centering\cardImg[height=0.8\textheight]{7j34up.jpg}{.8\textwidth}
\end{frame}

% \begin{frame}[label=why-2]
% \frametitle{In my talk I want to:}
% \begin{itemize}
% \item Explain how real-time 3D-PTV provides unique benefits, that are unavailable through other methods.
% \item Describe some of the potential applications of real-time 3D-PTV
% \item Present some examples of real-time 3D-PTV projects to illustrate their utility.
% \end{itemize}    
% \end{frame}

\begin{frame}[label=why-3]{We deal with complex flow problems in more challenging environments}
\begin{card}[``Simple'' problems are solved :) ]
    Simpler fluid mechanics cases are solved to a large extent, so we are facing more and more complex cases: 
    moving boundaries (e.g. aorta), 3D shapes (e.g. heart models), harsh environments and remote experiments (microgravity, ocean floor, UAV, AUV, biology, etc.)
\end{card}
\end{frame}

\begin{frame}[label=why-3a]{How many times \ldots }
\begin{itemize}
    \item Experiments prepared for days and weeks but we lost data due to faulty calibration? improper camera settings? not synced trigger? 
    \item Measured flow slightly different from the plan, because flow depends on boundary conditions, fluid properties (e.g. temperature, viscosity, density, coupling with other phases, etc.)
%\end{itemize}
%\end{frame}

%\begin{frame}[label=why-4]{How many times \ldots }
%\begin{itemize}
    \item missed the important flow event just before or after recording time, e.g. flow-mediated prey-predator interactions
    \item Flow has both long and short time scales? both high temporal resoluion and a long recording time? 
\end{itemize}
\end{frame}


% \begin{frame}{The Conceptual Advantage of Real-Time Information for Measurement Systems}

% \begin{itemize}
% \item Real-time information gives users a better understanding of the measurement process and enables them to make more informed decisions based on the data.
% \item Real-time information allows users to monitor the measurement process in real-time and detect any anomalies or issues as they occur: calibration/malfunction/out-of-sync
% \item Real-time information about uncertainties allows users better to understand the reliability and accuracy of the measured data.
% \item Real-time information about uncertainties can help users optimize the measurement process itself.
% \end{itemize}

% \end{frame}

\topicFramePrimary{What are the most prominent advantages of real-time 3D-PTV?}

\begin{frame}[label=why-5]
\frametitle{My best argument is that you can succeed where we failed before}
\centering \cardImg{Ron.jpg}{0.7\textwidth}
\begin{cardTiny}
Thanks to this technology, Dr. Ron Shnapp is a fresh tenure-track faculty at Ben Gurion University
\end{cardTiny}
\end{frame}


\begin{frame}[label=why-6]{Real-Time Information Enables Effective Troubleshooting and Maintenance}

\begin{itemize}
\item Real-time information allows for quick and effective troubleshooting and maintenance, which can help prevent further errors or inaccuracies (e.g. 3D calibration)
\item Early detection of issues is critical in safety-critical systems where even a small delay in detecting a problem can have serious consequences (e.g. reverse flow before critical heat flux)
\end{itemize}

\end{frame}

\begin{frame}[label=why-7]{Real-Time Information Improves Accuracy and Precision}

\begin{itemize}
\item Real-time information allows users to adjust the measurement process in real-time to improve accuracy and precision. (e.g. camera small shift)
\item Real-time information about uncertainties allows users to make more informed decisions about how to use the data and what level of confidence they can have in the results (e.g. agile: release fast, fail fast) 
\item Real-time monitoring of uncertainties can help users optimize the measurement process and reduce uncertainties to improve overall accuracy and precision.(e.g. improve focus, angle, aperture)
\end{itemize}
\end{frame}

% \begin{frame}[label=why-8]{Real-Time Information Enables Faster Decision-Making}
% \begin{itemize}
% \item Real-time output allows for faster decision-making by providing immediate feedback on the measured data. (e.g. throw wrong data and repeat expeirments)
% \item Real-time information is especially important in time-sensitive applications where quick decisions can make a significant difference
% \end{itemize}
% \end{frame}

\begin{frame}[label=why-9]{Real-Time Information Enables Process Control}

\begin{itemize}
\item Real-time output can be used to monitor and control processes in real time, allowing for adjustments to be made as needed to ensure optimal performance (future ideas, hopefully soon in chemical and bioreactors)
\item Real-time information is especially important in manufacturing and industrial applications where process control can impact quality, safety, and efficiency (e.g. painting, filtering, flotation, etc.)
\end{itemize}
\end{frame}

\begin{frame}[label=why-10]{Real-Time Information Enables Remote Monitoring}

\begin{itemize}
\item Real-time output can enable remote monitoring of measurement systems, allowing users to monitor and control systems from a distance (e.g. microgravity experiment in space)
\item Real-time information is especially important in applications where access to the measurement system may be limited or difficult (e.g. ocean floor, UAV, AUV)
\end{itemize}

\end{frame}

% \subsection{Why real-time 3D-PTV}

% \begin{frame}[label=why-11]{Key Advantages of Real-Time 3D-PTV}

% \begin{itemize}
% \item Immediate feedback on the velocity field - enables process control and optimization of parameters and setup.
% \item Enables faster decision-making and quick adjustments to be made as needed.
% \item Detailed enough insight into the fluid flow behavior for understanding the dynamics.
% \item Improve the accuracy and precision of the measured data by allowing for adjustments to be made in real-time to reduce uncertainties.
% \item Detect anomalies or issues as they occur, allowing for quick and effective troubleshooting and maintenance.
% \item To remotely monitor fluid flow behavior, especially important in applications where access to the measurement system may be limited or difficult (underwater, in-engine, in-flight).
% \end{itemize}
% \end{frame}

\begin{frame}[label=why-12]{We were lucky to capitalize on these benefits in the wind tunnel experiment}
\begin{columns}
\column{.5\textwidth}
    \cardImg{calibration_in_laser}{\textwidth}
    \cardImg{ptv_wind_tunnel_photo_3}{\textwidth}%
\column{.5\textwidth}
    \cardImg{traj_snapshot3}{\textwidth}%
    \cardImg{ptv_wind_tunnel_photo_4.png}{\textwidth}
\end{columns}
\end{frame}

% \begin{frame}{Immediate Feedback for Process Control and Optimization}

% \begin{itemize}
% \item Real-time 3D-PTV allows for immediate feedback on the velocity field, which can be used for process control and optimization.
% \item Faster decisions can be made to adjust the process as needed, ensuring optimal performance and efficiency.
% \end{itemize}

% \end{frame}

% \begin{frame}{Faster Decision-Making}

% \begin{itemize}
% \item Real-time 3D-PTV enables faster decision-making by providing immediate feedback on the fluid flow behavior, allowing for quick adjustments to be made as needed.
% \item Faster decisions can help prevent delays and improve overall efficiency in applications where time is critical.
% \end{itemize}

% \end{frame}

% % \begin{frame}{Detailed Insight into Fluid Dynamics}

% % \begin{itemize}
% % \item Real-time 3D-PTV can provide detailed insight into the fluid flow behavior, which can be used for research purposes and to improve understanding of fluid dynamics.
% % \item More detailed information can lead to discoveries and improvements in fluid dynamics research.
% % \end{itemize}
% \end{frame}

% \begin{frame}{Improved Accuracy and Precision}

% \begin{itemize}
% \item Real-time 3D-PTV can help improve the accuracy and precision of the measured data by allowing for adjustments to be made in real-time to reduce uncertainties.
% \item Real-time monitoring of uncertainties can help reduce errors and improve confidence in the measured data.
% \end{itemize}

% \end{frame}

% \begin{frame}{Effective Troubleshooting and Maintenance}

% \begin{itemize}
% \item Real-time 3D-PTV can help detect any anomalies or issues in the fluid flow behavior as they occur, allowing for quick and effective troubleshooting and maintenance.
% \item Early detection of issues is critical in safety-critical applications where delays can have serious consequences.
% \end{itemize}

% \end{frame}

% \begin{frame}{Remote Monitoring}

% \begin{itemize}
% \item Real-time 3D-PTV can be used to remotely monitor fluid flow behavior, which can be especially important in applications where access to the measurement system may be limited or difficult.
% \item Remote monitoring can allow for more efficient use of resources and improve safety in hazardous environments.
% \end{itemize}

% \end{frame}