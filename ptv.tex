\section{Basics of 3D-PTV and real-time 3D-PTV}
\topicFramePrimary{3D-PTV: essential factors and prerequisites}


% \begin{frame}
%     \frametitle{Feasibility and Difficulty}
%     \begin{itemize}
%     \item Discuss the technical feasibility of real-time 3D-PTV and the challenges that must be overcome to achieve it.
%     \item Highlight some of the key research areas that need to be addressed to make real-time 3D-PTV more feasible and efficient.
%     \item Provide examples of some of the tools and resources available to help researchers in this area.
%     \end{itemize}
% \end{frame}
    

\begin{frame}[label=ptv-1]{3D-PTV overview}
    \begin{multicols}{2}
    \begin{card}[What is 3D-PTV]
    3D-PTV is a technique that allows tracking of particles in three-dimensional space for improved fluid dynamics understanding, using the Lagrangian framework
    \end{card}
    \begin{card}[TL;DR]
    Multiple 2D projections of the particles are obtained and overlapped to create 3D reconstruction, than the particles' positions through sequential 3D clouds are linked in space. 
    \end{card}
    \end{multicols}
\end{frame}


%
\begin{frame}[label=ptv-2]{Shr\"{o}der and Schanz, Annu. Rev. Fluid Mech. 2023}
    \cardImg{shroder.jpeg}{\textwidth}
\end{frame}

% \begin{frame}[label=ptv-3]{This sculpture helped people to feel like a Lagrangian particle}
%     \centering\cardImg{art}{0.8\textwidth}
%     \vspace{-.3cm}
%     \begin{cardTiny}
%     ``Marianthe'' invited people inside turbulent forms to experience them as
%     if they were a particle borne along in the flow. Athena
%     Tacha (1985), \href{http://nautil.us/issue/15/turbulence/the-scientific-problem-that-must-be-experienced}{``Nautilus'' by Philip Ball}
%     \end{cardTiny}                                                          
% \end{frame}
% %

\begin{frame}[label=ptv-4]{Let's get a quick intro to 3D-PTV}
    \centering \cardImg[height]{ptv-scheme.png}{.75\textwidth}
\end{frame}


\begin{frame}[label=ptv-5]{Basic steps of 3D-PTV}
    \centering\cardImg[height]{ptv_steps}{.55\textwidth}
\end{frame}

% %%
% \begin{frame}[label=ptv-22]{Basic steps}
% \begin{multicols*}{2}
% 		\cardImg[height]{ptv_blocks}{.49\textwidth}
% 		\cardImg[height]{ptv_steps}{.49\textwidth}
% \end{multicols*}
% \end{frame}

\begin{frame}[label=ptv-6]{Epipolar geometry}
    \centering\cardImg[height=.8\textheight]{epipolar1}{0.8\textwidth}
\end{frame}

\begin{frame}[label=ptv-7]{Multi-view epipolar geometry}
    \centering\cardImg{epipolar}{\textwidth}
\end{frame}

\begin{frame}[label=ptv-8]{That's how it looks in the OpenPTV}
    \centering\cardImg[height=]{stereo_matching}{.7\textwidth}
\end{frame}

\begin{frame}[label=ptv-9a]{It strongly depends on 3D calibration}
    \centering\cardImg{calibration1}{\textwidth}
\end{frame}


\begin{frame}[label=ptv-9b]{We prefer multi-media non-linear calibration}
    \centering\cardImg[height]{multimedia}{.6\textwidth}
\end{frame}

\begin{frame}[label=ptv-10]{Four frames sliding tracking (robust and fast)}
    \centering\cardImg[height]{tracking}{1\textwidth}
\end{frame}

\begin{frame}[label=ptv-11]{Tracking in 2D space vs 3D space}
    \centering\cardImg[height=.8\textheight]{tracking_in_image_space_projection.png}{.6\textwidth}
\end{frame}


\begin{frame}[label=ptv-12]{Tracking in 2D or 3D space}
% \begin{multicols*}{2}

\centering \cardImg[height]{tracking_2d_vs_3d}{.6\textwidth}

% \cardImg[height]{tracking_in_3d_space}{.45\textwidth}

% \end{multicols*}
\end{frame}


% \begin{frame}[label=ptv-13]{Tracking in 3D space }
% 	\centering
% \end{frame}


% \begin{frame}[label=ptv-14]{Tracking in both 2D and 3D}
% 	\centering\cardImg[height=.8\textheight]{image_space_tracking}{.7\textwidth}
% \end{frame}

			

\begin{frame}[label=ptv-15]{The results are Lagrangian trajectories, $X(a,t)$}
    %\begin{multicols}{2}
    \centering
    \cardImg[height]{lagrangian_trajectory}{.9\textwidth}
    % \cardImg{particle_trajectories}{.39\textwidth}
    % \end{multicols}
\end{frame}


\begin{frame}[label=ptv-16]{Two-frame PTV is somewhere between Eulerian and Lagrangian}
    \centering\cardImg{eulerian_vs_lagrangian}{.9\textwidth}
\end{frame}

% \begin{frame}[label=ptv-8]
% \frametitle{Time-resolved is a tautology, real-time is real :) }
% \begin{enumerate}
% \item Time-resolved PTV is a tautology - to track objects we have to resolve their position in time :) 
% \item Real-time is not at the velocity of the particle, but during the experiment: we get trajectories on the screen when the flow is moving (not the same flow)
% \end{enumerate}
% \end{frame}

